\section{Conclusion}

// J'ai remarquer que :

// * Les test unitaire de Eigen sont très long ce qui fait qui ne mon pas été utile
pour teseter mes modifications

// * Les test unitaire de Eigen on des bug ce qu'il fait qu'il faut les lancer plusieurs
fois pour qu'il fonctionne (c'est un bug connue) (trouver l'issus)

// * J'ai remarquer que plus on code en MIPP dans Eigen plus le temps de compilation est
élever. C'est très certainement du au templates mais je ne peut pas dire si cela viens
des templates dans MIPP ou dans Eigen ou les 2.

// Les connaissances que j'ai acquis lors de mes étude que j'ai utiliser au cours du
stage :

// * Les test de non regression que j'ai du implementer pour verifier que mes
implémentation été bonne. (AL / PdP / PLE / SLAM4) (Le GL)

// * Le fait de refactorisé du code (utilisation des 5 principe\dots) (AL / PdP)

// * L'utilisation des intransics et le calcule vectoriel (vue en PAP)

// * Les base en C++ vue un petit-peu en OS dans nahos mais aussi dans mon projet PdP\dots

// * La comprehension du monde de la recherche grace à l'UE initiation recherche.

// Ce que ma apporter le stage :

// * une mayeur comprehension des cast, conversion des type de base et des vecteurs

// * Utilisation avancer des templates en c++

// * des Utilisation différentes des vecteurs, plusieurs vision de leur utilisation.
(taille fix ou non)

// * une mayeur comprehension du monde de la recherche

