\section{Conclusion}

Le premier objectif est en partie rempli car j'ai implémenté une partie des opérations et
j'ai bien vue les différences entre la manipulation des vecteurs dans \Eigen et \MIPP.
Par-contre je n'ai pas pus faire une campagne d'évaluation des performances ni le tester
d'\Eigen sur l'architecture \emph{Risc-V} car toute les operations ne sont pas codé en
\emph{code MIPP}.

\subsection{Ce que j'ai remarquer}

\begin{itemize}
  \item Les test unitaire de \Eigen sont très long ce qui fait qui ne mon pas été utile
  pour teseter mes modifications.
  \item Les test unitaire de \Eigen on des bug ce qu'il fait qu'il faut les lancer
  plusieurs fois pour qu'il fonctionne.
  \item J'ai remarquer que plus je codé en \emph{code MIPP} dans \Eigen plus le temps de
  compilation été élevé. C'est très certainement du au templates mais je ne peut pas dire
  si cela viens des templates dans \MIPP ou dans \Eigen ou les deux.
\end{itemize}

Grâce à mes tests et mes implémentations \MIPP dans \Eigen j'ai pu lister des operations
qu'il manque dans \MIPP :

\begin{table}[H]
  \centering
  \caption*{\texttimes: absent de \MIPP \checkmark: present dans \MIPP}
  \begin{tabularx}{\linewidth}[H]{|m{.238\linewidth}|m{.1205\linewidth}|m{.0872\linewidth}|m{.0705\linewidth}|m{.1594\linewidth}|X|} %m{.108\linewidth}
  % \begin{tabularx}{\linewidth}[H]{|l|c|c|c|c|c|}
    \hline
                     & \textbf{AVX512} & \textbf{AVX2} & \textbf{AVX} & \textbf{SSE4.1/4.2} & \textbf{SSE2/3} \\
    \hline
    add<int32_t>    & \checkmark      & \checkmark    & \texttimes   & \checkmark          & \checkmark      \\
    \hline
    sub<int32_t>    & \checkmark      & \checkmark    & \texttimes   & \checkmark          & \checkmark      \\
    \hline
    mul<int32_t>    & \checkmark      & \checkmark    & \texttimes   & \checkmark          & \texttimes      \\
    \hline
    orb<int8_t>     & \checkmark      & \checkmark    & \texttimes   & \checkmark          & \checkmark      \\
    \hline
    xor<int8_t>     & \checkmark      & \checkmark    & \texttimes   & \checkmark          & \checkmark      \\
    \hline
    and<int8_t>     & \checkmark      & \checkmark    & \texttimes   & \checkmark          & \checkmark      \\
    \hline
    cmpneq<int16_t> & \texttimes      & \checkmark    & \texttimes   & \texttimes          & \texttimes      \\
    \hline
    cmpneq<int8_t>  & \texttimes      & \checkmark    & \texttimes   & \texttimes          & \texttimes      \\
    \hline
  \end{tabularx}
  \caption{Abstractions \MIPP non implementer}
\end{table}

Pour exécuter le code en \emph{AVX512} j'ai utilisé \emph{PlaFRIM}.

\subsection{Les connaissances que j'ai acquis lors de mes études que j'ai utilisé au
cours du stage}

\begin{itemize}
  \item Les tests de non regression que j'ai du implementer pour verifier que mes
  implémentations été conforme à la version précédente. J'ai obtenu c'est compétences de
  des lors des cours d'Architecture Logiciel (AL), de Projet de Programmation (PdP) et dans
  un cours de BTS (SLAM4).
  \item La refactorisation de code et l'utilisation des 5 principe solides. Que j'ai
  appris en cours d'Architecture Logiciel (AL) et de Projet de Programmation (PdP).
  \item L'utilisation des intransics et le calcule vectoriel que j'ai vue en Programmation
  sur Architecture Parallèles (PAP).
  \item Les base en C++ que j'ai vue un petit-peu dans nachos en tp Système
  d'Exploitation(SE) mais aussi dans le projet PdP de mon équipe.
  \item La comprehension du monde de la recherche grâce à l'UE initiation recherche de L3.
\end{itemize}

\subsection{Ce que ma apporter le stage}

\begin{itemize}
  \item Une mayeur comprehension des cast, des conversions des type de base et de la
  manipulation des des vecteurs.
  \item Une nouvelle expérience de développement dans un projet opensource qui est très
  utiliser.
  \item L'utilisation avancé des templates en c++.
  \item Des utilisation différentes des vecteurs avec plusieurs vision de leur
  utilisation.
  \item Une mayeur comprehension du monde de la recherche.
\end{itemize}
