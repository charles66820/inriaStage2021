\section{Objet du stage / travaille demandé}

// faire le portage de Eigen sur MIPP

// Eigen est une lib\dots elle est capable de faire des calcule vectoriser mais pour
cela elle a une implementation par architectures.

// MIPP est une lib\dots MIPP est un sous partie d'AFF3CT (A Fast Forward Error Correction
Toolbox) qui\dots.

// MIPP permet de faire des operations vectoriel\dots

// Le premier objectif est donc d'ajouter, avec l'interface MIPP, une implementation
vectoriel des fonctions élémentaires de Eigen pour que le support des architectures sois
automatique.

// Le second objectif est de faire une campagne d'évaluation des performances (avant /
après MIPP)

// Le dernier objectif est de tester Eigen sur l'architecture Risc-V qui n'est pas encore
présent den Eigen et évaluer les performances sur simulateur.

// L'objectif a long terme est de pouvoirs garder uniquement l'implementation MIPP et
supprimer les autre.

// Un des intérêt de faire ce portage et de permettre le support de future architectures
sans avoir a l'ajouter mais juste en mettant a jour MIPP.

