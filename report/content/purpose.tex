\section{Objet du stage / travail demandé}

L'objectif de mon stage est de faire le portage de \Eigen sur \MIPP.

\subsection{Les bibliothèques}

\Eigen est une bibliothèque open source écrite en C++11 très utilisé. Elle permet
de faire de l'algèbre linéaire, de la manipulation de matrices, de vecteurs,
de solveurs numérique et \textcolor{red}{related algorithms}. Pour accélérer les calculs,
Eigen utilise une vectorisation explicite. Il y a donc une implémentation par
architecture. Elle est principalement développée au centre Inria de l'université de
Bordeaux et est au cœur d'autres bibliothèques comme TensorFlow.

% définition d'intrinsic ?

\MIPP est une bibliothèque open source écrite en C++11 qui fournit une abstraction
unique pour les fonctions intrinsic\footnote{intrinsic : une instruction SIMD} (SIMD
\footnote{SIMD (Single Instruction on Multiple Data) est une architecture parallèle
qui permet à une intrinsic de faire simultanément des opérations sur plusieurs données
(un ou plusieurs vecteurs) et produire plusieurs résultats}) de plusieurs architectures.
Elle fonctionne actuellement pour les architectures SSE, AVX, AVX512, ARM NEON (32bits and
64bits). Elle supporte les nombres flottants de précision simple et double ainsi que les
entiers signés et codés sur 64, 32, 16 et 8 bits. Son objectif est d'écrire une seule fois un
code qui utilise les fonctions de \MIPP, j'appellerai ce code \emph{code MIPP} dans
la suite du rapport, sans avoir à écrire un code d'intrinsic spécifique pour chaque
architecture. \MIPP fournit automatiquement, à partir d'un \emph{code MIPP}, les
bonnes intrinsic pour une architecture spécifique. \MIPP est une sous partie
d'\emph{AFF3CT} (A Fast Forward Error Correction Toolbox) qui est une bibliothèque et un
simulateur qui est dédié au \emph{Forward Error Correction (FEC or channel coding)}. Elle
est également écrite en C++.

\subsection{Les objectifs}

Le premier objectif est donc d'ajouter une nouvelle implémentation vectorielle, en
\emph{code MIPP}, des fonctions élémentaires de \Eigen. Ceci afin de permettre que le
support des différentes architectures soit automatique.

Le second objectif est de faire une campagne d'évaluation des performances pour voir
s'il y a une différence entre \Eigen sans l'implémentation en \emph{code MIPP} et
\Eigen avec l'implémentation en \emph{code MIPP}. Il n'y a pas de raison d'avoir
de meilleures performances, mais il peut y avoir une légère dégradation.

Le dernier objectif est de tester \Eigen sur l'architecture \emph{Risc-V} qui n'est
pas encore présent dans \Eigen et évaluer les performances sur simulateur.

L'objectif à long terme est de pouvoir garder uniquement l'implémentation en
\emph{code MIPP} qui remplace les autres implémentations explicites.

Les intérêts de ce portage sont :
\begin{itemize}
  \item la réduction du nombre de lignes de code et de la complexité du code.
  \item permettre le support de futures architectures sans avoir à refaire toute une
  implémentation explicite, mais tous simplement en mettant à jour \MIPP.
\end{itemize}
