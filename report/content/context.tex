\section{Contexte}
% 2 mois en M1
J'ai effectué mon stage dans l'équipe-projet STORM à Inria (Institut National de
Recherche en Informatique et en Automatique). Inria est un établissement public à
caractère scientifique et technologique.
L'équipe STORM (STatic Optimizations and Runtime Methods) travaille dans le domaine du
calcul haute performance, le HPC (High Performance Computing). Plus précisément sur
de  nouvelles interfaces de programmation et langages pour exprimer le parallélisme
hétérogène et massif. Le but est de fournir des abstractions des architectures tous en
garantissant la compatibilité haute performance aussi qu'une bonne efficacité de calcul
et énergétique.
L'équipe est constituée de chercheurs, d'enseignants-chercheurs, d'ingénieurs de recherche,
de doctorants, et de stagiaires.
L'équipe à une culture informatique liée à son domaine de recherche :

\begin{itemize}
  \item Langues de haut niveau spécifiques à un domaine. (High level domain specific languages)
  \item Les Runtime hétérogènes, les plates-formes multi-cœurs. (Runtime systems for heterogeneous, manycore platforms)
  \item Des outils d'analyse et de retour de performance. (Analysis and performance feedback tools)
\end{itemize}

Les membres permanents ont tous fait des études en informatique et ont fait une thèse en
informatique dans le domaine du HPC. Ils ont des compétences diverses : compilation, runtime,
architecture matérielle, langage bas niveau, langages parallèles, gestion de tâches, ordonnancement\dots

Le matériel et les logiciels mis à ma disposition sont un ordinateur portable avec Linux
et la possibilité d'installer les applications dont j'ai besoin (vscode, \LaTeX\dots), un
accès à la plateforme de calcul \emph{PlaFRIM} qui fournit un ensemble de machines (nœuds)
aux chercheurs, entreprise SME et étudiants qui en ont besoin. \emph{PlaFRIM} comporte une
multitude de nœuds avec des architectures différentes (SSE, AVX, ARM NEON\dots).
