\section{Contexte}
% 2 mois en M1
J'ai effectuer mon stage dans l'équipe-projet STORM à Inria (Institut National de
Recherche en Informatique et en Automatique). Inria est un établissement public à
caractère scientifique et technologique.
L'équipe STORM (STatic Optimizations and Runtime Methods) travaille dans le domain du
calcule haute performance, le HPC (High Performance Computing). Plus précisément sur
de  nouvelles interfaces de programmation et langages pour exprimer le parallélisme
hétérogène et massif. Le but est de fournir des abstractions des architecture tous en
garantissent la compatibilité haute performance aussi qu'une bonne efficacité de calcule
et énergétique.
L'équipe est constitué de chercheurs, d'enseignants-chercheurs, d'ingénieurs de recherche,
de doctorants, et de stagiaires.
L'équipe à une culture informatique lié à sont domain de recherche :

\begin{itemize}
  \item Langues de haute niveau spécifiques à un domaine. (High level domain specific languages)
  \item les Runtime hétérogènes, les plates-formes multi cœurs. (Runtime systems for heterogeneous, manycore platforms )
  \item Des outils d'analyse et de retour de performance. (Analysis and performance feedback tools)
\end{itemize}

Les membres permanents on tous fait des étude en informatique et on eu une thèse en
informatique dans le domain du HPC. Ils on des compétence diverse : compilation, runtime,
architecture materiel, language bas niveau, language parallèles, gestion de tâches\dots

Le materiel et les logiciels mis à ma disposition sont un ordinateur portable avec linux
et la possibilité d'installer les application d'on j'ai besoin (vscode, \LaTeX\dots), un
accès à la platforms de calcule \emph{PlaFRIM} qui fournir un ensemble de machine (noeuds)
au chercheurs, entreprise SME et étudiants qui en on besoin. \emph{PlaFRIM} comporte une
multitude de noeuds avec des architectures différentes (SSE, AVX, ARM NEON\dots).
