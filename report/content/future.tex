\section{Futur / ce qui reste à faire}

Il reste à implémenter toutes les opérations qui ne sont pas encore en \emph{code MIPP} et
faire un \textit{repliage} qui fonctionne correctement.

Il serait intéressant d'ajouter les évolutions suivantes à \MIPP :

\begin{itemize}
  \item Ajouter le type boolean (\verb|bool|), le type half (\emph{HF} ou
  \verb|_m512h| en \emph{AVX512}) et les nombres complexes.
  \item Ajouter des types non signés (\verb|uint8_t|, \verb|uint16_t|\dots)
  \item Faire en sorte que les \emph{Reg2} soit affichable en implémentant l'opérateur \emph{to string}.
  \item Ajouter la correspondance int8_t et char pour permettre à l'utilisateur de faire
  des casts avec char.
  \item Faire en sorte que l'abstraction \verb|testz| fonctionne sur toutes les architectures.
  \item Peut-être ajouter les opérations pour le Reg2 mais ce n'est peut-être pas dans
  l'esprit de \emph{MIPP}
  \item Trouver ce qui cause un temps de compilation excessif et le réduire.
\end{itemize}

\begin{table}[H]
  \centering
  \caption*{En \textcolor{darkGreen}{vert} les types qui fonctionnent et en \textcolor{orange}{orange}
    les types qui ne fonctionnent pas dans \emph{MIPP}.}
  \begin{tabular}[H]{|l|l|l|}
    \hline
    \textbf{Type standard}          & \textbf{type de base}        & \textbf{type unsigned}        \\
    \hline
    \textcolor{darkGreen}{int8_t}  & \textcolor{orange}{char}     & \textcolor{orange}{uint8_t}  \\
    \hline
    \textcolor{darkGreen}{int16_t} & \textcolor{darkGreen}{short} & \textcolor{orange}{uint16_t} \\
    \hline
    \textcolor{darkGreen}{int32_t} & \textcolor{darkGreen}{int}   & \textcolor{orange}{uint32_t} \\
    \hline
    \textcolor{darkGreen}{int64_t} & \textcolor{darkGreen}{long}  & \textcolor{orange}{uint64_t} \\
    \hline
    \textcolor{darkGreen}{float}    & \----                        & \----                         \\
    \hline
    \textcolor{darkGreen}{double}   & \----                        & \----                         \\
    \hline
    \textcolor{orange}{bool}        & \----                        & \----                         \\
    \hline
  \end{tabular}
  \caption{Types compatibles avec \MIPP}
\end{table}
