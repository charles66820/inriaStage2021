\section{Future / ce qui reste à faire}

//  Il reste a implementer tous le reste

//  Faire un repliage qui fonctionne

//  Modifier MIPP pour ajouter tous ce qu'il manque\dots

// Evolution possible pour MIPP :

// * Ajout du type bool, d'un type h  et complex (present en AVX512)

// * ajouter la correspondance int8\_t et char (pour le cast)

// * ajout des type unsigned (uint8\_t, uint16\_t\dots)

// * make Reg2 printable

// * peut-être ajouter les operations pour le Reg2 mais ce n'ai peut-être pas dans
l'esprit de MIPP

// * faire en sorte que le testz fonctionne avec toute les architecture

// * voire le temps de compilation

\begin{table}[H]
  \centering
  \caption*{En \textcolor{darkGreen}{vert} les type qui fonctionne, En \textcolor{orange}{orange}
    les type qui ne fonctionne}
  \begin{tabular}[H]{|m{.21\linewidth}|m{.18\linewidth}|m{.21\linewidth}|}
    \hline
    \textbf{Type standard}          & \textbf{type de base}        & \textbf{type unsigned}        \\
    \hline
    \textcolor{darkGreen}{int8\_t}  & \textcolor{orange}{char}     & \textcolor{orange}{uint8\_t}  \\
    \hline
    \textcolor{darkGreen}{int16\_t} & \textcolor{darkGreen}{short} & \textcolor{orange}{uint16\_t} \\
    \hline
    \textcolor{darkGreen}{int32\_t} & \textcolor{darkGreen}{int}   & \textcolor{orange}{uint32\_t} \\
    \hline
    \textcolor{darkGreen}{int64\_t} & \textcolor{darkGreen}{long}  & \textcolor{orange}{uint64\_t} \\
    \hline
    \textcolor{darkGreen}{float}    & \----                        & \----                         \\
    \hline
    \textcolor{darkGreen}{double}   & \----                        & \----                         \\
    \hline
    \textcolor{orange}{bool}        & \----                        & \----                         \\
    \hline
  \end{tabular}
  \caption{Type compatible avec MIPP}
\end{table}
