\section{Travaille réaliser}

\subsection{Défrichage / démarrage ?}

Pour prend en main \Eigen et \MIPP j'ai commancer par implementer un produit
vecteur matrice avec les deux bibliothèques.

Ensuite j'ai chercher ou ce trouvé les implémentations vectoriel explicite dans \Eigen
et comment elles sont implémentées.

Chaque architectures sont implémentées en 4 fichiers. Le premier \emph{PacketMath.h}
définit \hyperref[typeEigen]{les types} que \Eigen utilise pour représenté les
vecteurs et les implémentations des operations vectoriel pour chacun d'entre eux.
Il y a une implémentation générique de ce fichier pour les type scalaire. Le second
\emph{TypeCasting.h} définit les conversion entre les différant \hyperref[typeEigen]{types
vectoriel d'\Eigen}. C'est conversions concerne un peut toute les combinaison
de vecteurs de type entier, flottant, double, booléen, half\footnote{type present dans
\Eigen} et bfloat16\footnote{le type bfloat16 est décrit
\href{https://en.wikipedia.org/wiki/Bfloat16_floating-point_format}{ici}. Il y a une
implémentation présent dans \Eigen}. Le troisièmes \emph{MathFunctions.h}
implémente les operations mathématique non élémentaire comme : \texttt{ log, log2, log1p,
expm1, exp, sin, cos, sqrt, rsqrt, reciprocal, tanh, frexp, ldexp}. Le quatrième
\emph{Complex.h} définit \hyperref[typeEigen]{les types vectoriel d'\Eigen} et
les operations sur les nombre complex, simple et double precision.

\subsection{Les types représentent les vecteurs}
\Eigen et \MIPP représentent les vecteurs différemment voyons cela.

\label{typeEigen}
\subsubsection{La representation de vecteur dans \Eigen}
{
  Les vecteurs sont de taille fix et représenté par des types. Ces représentation sont
  dépendent de l'architecture visé. Ces types sont nommés \emph{Packet} suivi d'un nombre.
  Ce nombre corresponde au nombre d'éléments que le vecteur peut contenir. Les type ce
  termine par une ou plusieurs letter qui corresponde au type des éléments du vecteur.
  Par example un vecteur qui contient quatre flottant sera nommé \emph{Packet4f} et
  correspondra au type AVX \emph{__m128} car un flottant et codé sur 32bits donc
  $32\cdot 4=128$. Il y as donc des représentation de différant tailles pour les entiers,
  les flottants (simple et double precision), les nombres complex\dots
  \Eigen et donc capable d'utilisé des taille de vecteur différente en même temps.
}

\subsubsection{La representation de vecteur dans \MIPP}
{
  Les vecteurs sont de taille variable en fonction de l'architecture visé. \MIPP permet
  d'obtenir la taille des vecteurs qu'il manipule. Le \emph{code MIPP} dois respecté
  Le fait que les vecteurs on une taille variable. Il y a donc que 2 type de vecteurs
  \emph{reg} et \emph{reg2} qui fait la moitiés de la taille de \emph{reg}.
  \MIPP n'est donc pas capable d'utilisé différente taille de vecteur à la fois.
}

\subsection{Ajoute d'une nouvelle architecture \MIPP}

Pour ajouter la nouvelle architecture \MIPP j'ai créé un nouveaux dossier \verb|MIPP|
dans le dossier \verb|Eigen/src/Core/arch/|. J'ai ensuite créé le 4 fichiers
\emph{PacketMath.h}, \emph{TypeCasting.h}, \emph{MathFunctions.h} et \emph{Complex.h}
en copient le contenu des fichiers des architecture \emph{SSE}, \emph{AVX} et \emph{AVX512}
que j'ai adapté pour que tous fonctionne. Je me suis basé sur c'est trois architecture
car c'est celles qui ce trouver sur mon ordinateur. Il à été aisé de les faire fonctionné
ensemble car lorsque l'architecture \emph{AVX} est definit l'architecture \emph{SSE} l'ai
aussi.
Une fois l'architecture ajouté j'ai modifier le fichier
\verb|Eigen/src/Core/util/ConfigureVectorization.h| pour y ajouter la définition de la
macro \emph{EIGEN_VECTORIZE_MIPP} dans le cas où \emph{__MIPP__} est définit à la
compilation. Cette macro ce définit en plus des macro de l'architecture actuel
\emph{EIGEN_VECTORIZE_AVX}, \emph{EIGEN_VECTORIZE_SSE}\dots
Pour finir j'ai modifier le fichier \verb|Eigen/Core| pour charger les quatre fichiers
de la nouvelle architecture \MIPP. Le chargement de l'architecture \MIPP est prioritaire
par rapport aux autre.

\subsection{Types et operations \Eigen ?}

J'ai listé les différents types \Eigen définit ainsi que leur operations pour les
architectures \emph{SSE}, \emph{AVX}, \emph{AVX2} et \emph{AVX512}.

Les types :

\begin{table}[H]
  \label{eigenTypesTable}
  \centering
  \begin{tabular}[H]{|l|l|l|l|}
    \hline
    \textbf{SSE} \verb|__m128| & \textbf{AVX} \verb|__m256| & \textbf{AVX2} \verb|__m256| & \textbf{AVX512} \verb|__m512|\\
    \hline
    Packet4f     & Packet8f     & Packet4l      & Packet16f       \\
    \hline
    Packet2d     & Packet4d     &               & Packet8d        \\
    \hline
    Packet4i     & Packet8i     &               & Packet16i       \\
    \hline
    Packet16b    & Packet8h     &               & Packet16h       \\
    \hline
                 & Packet8bf    &               & Packet16bf      \\
    \hline
  \end{tabular}
  \caption{Les type vectoriel \Eigen par architecture}
\end{table}

Les opérations :

// TODO: (tableau avec toute les fonctions)?

\subsection{Les tests}

Avant de commancer l'implémentation en \emph{code MIPP} j'ai lancer les tests \Eigen.
En lancent ces tests j'ai remarquer qu'il sont très long à compiler et à ce lancer, de
plus il ne fonctionne pas tous à tout les coups. Les tests vont un peut plus vite sur
une machine plus puissante mais cela reste très long.

Pour que je puis tester efficacement et rapidement j'ai donc implémenté des tests de non
regression. Pour ces tests j'ai copier les fonctions actuel dans un nouveaux fichier et
je suffixés ces fonctions par \emph{_old}.

J'ai ensuite implémenté les tests de non regression pour toutes les operations présent
dans \emph{PacketMath.h}. Ces test fonctionne pour les architectures \emph{SSE}, \emph{AVX},
\emph{AVX2} et quelque operations en \emph{AVX512}.

Mes test de non régression son capable d'afficher le contenu des vecteurs lorsque il y a
une différence entre le résultat de la nouvelle version et celui de l'ancienne version
de l'operation.
Ils sont aussi capable d'afficher le contenu des vecteurs en binaire et de dire, sans
arrêter les test, quant le \emph{code MIPP} appelé une fonction \MIPP sur un type qui
n'est pas encore supporter par \MIPP.

\subsection{l'implémentation}

Dans un premier temps j'ai implémenté seulement les operations \emph{SSE}, \emph{AVX} et
\emph{AVX2}.
J'ai commancé par l'implémentation de l'opération \emph{pset1} qui rempli un vecteur avec
la même valeur à chaque case. Dans le \hyperref[eigenTypesTable]{tableau des types \Eigen}
on vois que les types \emph{SSE} font 128 bits et les type \emph{AVX} font 256 bits. Je
suis donc partie sur une mauvaise piste en convertissant les \emph{Packet4f},
\emph{Packet2d} et\emph{Packet4i} en \emph{reg2} et les \emph{Packet8f}, \emph{Packet4d}
et \emph{Packet8i} en \emph{reg}. Le problème est que \MIPP ne support qu'une seul taille
de vecteur à la fois donc il ne peut pas faire des opérations sur les \emph{reg2} mais
que sur les \emph{reg}. Ce problème ma amené à devoir transformer mes vecteurs \emph{reg2}
en \emph{reg} ce qui ce fait avec la fonction \emph{combinate()} et \emph{low()}, pour
respectivement fusionné deux \emph{reg2} vecteurs en un vecteur \emph{reg} et récupérer
la premier moitiés \emph{reg2} d'un vecteur \emph{reg}. Toute c'est transformation amené
à deux problème principaux. Le premier est que on a un surcoûts supplémentaire a chaque
appelle. Le second est que cette stratégie ne fonctionne pas avec \emph{AVX512} qu'on a
mis de côté pour l'instant. Néanmoins ces premier implémentations fonctionne et passe
mes tests.

Pour palier ce problème j'ai donc ajouté un système de conversion. Pour cela j'ai du pour
chaque architectures ajouté des cast\footnote{conversion d'un type à un autre} simple et
des cast en intrinsic pour les changement de taille, Par exemple un vecteur de 128 bits
que je dois convertir en vecteur de 512 bits et inversement.
Ces conversion peuvent amené a une perte de performance.

TODO: (Benchmark here ?)

J'ai ensuite continué l'implémentation en \emph{code MIPP} de plusieurs opérations pour
chaque types \emph{SSE} et \emph{AVX} quelle support. Ce qui ma amené à avoir plusieurs
fois la même implémentation \MIPP avec seulement le type qui change. Par exemple
l'operation \emph{pset1<Packet4f>()} et l'operation \emph{pset1<Packet4i>}.

Dans un second temps le but à été de \textit{replier} les fonctions pour en avoir qu'une
seul. Mais je suis tombés sur un problème, le comportement par défaut est une operation
scalaire. En effet \Eigen permet d'activer ou non la vectorisation. Lorsque la
vectorisation et désactiver se sont les même operations qui sont appelé mais dans ce cas
se sont les operation par défaut qui sont appelé. Et dans le cas où la vectorisation et
activer on tombe sur un cas spécifique, les cas avec les \emph{Packet} qui sont précisé
à l'appelle. On ce retrouve donc avec 2 cas générique, un pour les scalaires
et un pour les vecteurs (\emph{Packet}). Cette situation amené à un état compliqué qui
reste encore à résoudre.

Je vois deux solutions qui fonctionneraient peut-être :

\begin{itemize}
  \item finir de tous implémenté puis faire en sorte que tous les \emph{Packet}
  corresponde au type \emph{reg} de \MIPP ce qui permettrai d'avoir qu'un seul cas
  spécifique qui correspondrai au \emph{code MIPP}. Cela nous permettrai en plus de ne
  plus avoir besoin du système de conversion.
  \item ajouté des adaptateur d'objet pour avoir une arborescence et utiliser le
  polymorphisme pour avoir un seul cas spécifique qui correspondrai au \emph{code MIPP}.
\end{itemize}
