\section{Travail réalisé}

\subsection{Défrichage}

Pour prendre en main \Eigen et \MIPP j'ai commencé par implémenter un produit
vecteur matrice avec les deux bibliothèques.

Ensuite j'ai cherché ou se trouvait les implémentations vectorielles explicites dans \Eigen
et comment elles sont implémentées.

Chaque architecture est implémentée en 4 fichiers. Le premier \emph{PacketMath.h}
définit \hyperref[typeEigen]{les types} que \Eigen utilise pour représenter les
vecteurs et les implémentations des opérations vectorielles pour chacun d'entre eux.
Il y a une implémentation générique de ce fichier pour les types scalaires. Le second
\emph{TypeCasting.h} définit les conversions entre les différents \hyperref[typeEigen]{types
vectoriel d'\Eigen}. Ces conversions concernent un peu toute les combinaisons
de vecteurs de type entier, flottant, double, booléen, half\footnote{type présent dans
\Eigen} et bfloat16\footnote{le type bfloat16 est décrit
\href{https://en.wikipedia.org/wiki/Bfloat16_floating-point_format}{ici}. Il y a une
implémentation présente dans \Eigen}. Le troisième \emph{MathFunctions.h}
implémente les opérations mathématiques non élémentaires comme : \texttt{ log, log2, log1p,
expm1, exp, sin, cos, sqrt, rsqrt, reciprocal, tanh, frexp, ldexp}. Le quatrième
\emph{Complex.h} définit \hyperref[typeEigen]{les types vectoriels d'\Eigen} et
les opérations sur les nombres complexes, simples et doubles précisions.

\subsection{Les types représentant les vecteurs}
\Eigen et \MIPP représentent les vecteurs différemment, voyons cela.

\label{typeEigen}
\subsubsection{La représentation des vecteurs dans \Eigen}
{
  Les vecteurs sont de taille fixe et représentés par des types. Ces représentations sont
  dépendantes de l'architecture visée. Ces types sont nommés \emph{Packet} suivis d'un nombre.
  Ce nombre correspond au nombre d'éléments que le vecteur peut contenir. Les types se
  terminent par une ou plusieurs lettres qui correspondent au type des éléments du vecteur.
  Par exemple, un vecteur qui contient quatre flottants sera nommé \emph{Packet4f} et
  correspondra au type AVX \emph{__m128} car un flottant est codé sur 32 bits donc
  $32\cdot 4=128$. Il y a donc des représentations de différentes tailles pour les entiers,
  les flottants (simple et double précision), les nombres complexes\dots\space
  \Eigen et donc capable d'utiliser des tailles de vecteur différentes en même temps.
}

\subsubsection{La représentation des vecteurs dans \MIPP}
{
  Les vecteurs sont de tailles variables en fonction de l'architecture visée. \MIPP permet
  d'obtenir la taille des vecteurs qu'il manipule. Le \emph{code MIPP} doit respecter
  le fait que les vecteurs ont une taille variable. Il n'y a donc que 2 types de vecteurs :
  \emph{reg} et \emph{reg2} qui fait la moitié de la taille de \emph{reg}.
  \MIPP n'est donc pas capable d'utiliser différentes tailles de vecteur à la fois.
}

\subsection{Ajout d'une nouvelle architecture \MIPP}

Pour ajouter la nouvelle architecture \MIPP, j'ai créé un nouveau dossier \verb|MIPP|
dans le dossier \verb|Eigen/src/Core/arch/|. J'ai ensuite créé les 4 fichiers
\emph{PacketMath.h}, \emph{TypeCasting.h}, \emph{MathFunctions.h} et \emph{Complex.h}
en copiant le contenu des fichiers des architectures \emph{SSE}, \emph{AVX} et \emph{AVX512}
que j'ai adapté pour que tout fonctionne. Je me suis basé sur ces trois architectures,
car se sont celles qui se trouvent sur mon ordinateur. Il a été aisé de les faire fonctionner
ensemble, car lorsque l'architecture \emph{AVX} est définie, l'architecture \emph{SSE} l'est
aussi.
Une fois l'architecture ajoutée, j'ai modifié le fichier
\verb|Eigen/src/Core/util/ConfigureVectorization.h| pour y ajouter la définition de la
macro \emph{EIGEN_VECTORIZE_MIPP} dans le cas où \emph{__MIPP__} est défini à la
compilation. Cette macro se définit en plus des macros de l'architecture actuelle
\emph{EIGEN_VECTORIZE_AVX}, \emph{EIGEN_VECTORIZE_SSE}\dots\space
Pour finir, j'ai modifié le fichier \verb|Eigen/Core| pour charger les quatre fichiers
de la nouvelle architecture \emph{MIPP} Le chargement de l'architecture \MIPP est prioritaire
par rapport aux autres.

\subsection{Types et opérations \Eigen}

J'ai listé les différents types \Eigen définis ainsi que leurs opérations pour les
architectures \emph{SSE}, \emph{AVX}, \emph{AVX2} et \emph{AVX512}.

Les types :

\begin{table}[H]
  \label{eigenTypesTable}
  \centering
  \begin{tabular}[H]{|l|l|l|l|}
    \hline
    \textbf{SSE} \verb|__m128| & \textbf{AVX} \verb|__m256| & \textbf{AVX2} \verb|__m256| & \textbf{AVX512} \verb|__m512|\\
    \hline
    Packet4f     & Packet8f     & Packet4l      & Packet16f       \\
    \hline
    Packet2d     & Packet4d     &               & Packet8d        \\
    \hline
    Packet4i     & Packet8i     &               & Packet16i       \\
    \hline
    Packet16b    & Packet8h     &               & Packet16h       \\
    \hline
                 & Packet8bf    &               & Packet16bf      \\
    \hline
  \end{tabular}
  \caption{Les types vectoriels \Eigen par architecture}
\end{table}

Les opérations :

// TODO: (tableau avec toutes les fonctions)?

\subsection{Les tests}

Avant de commencer l'implémentation en \emph{code MIPP}, j'ai lancé les tests \emph{Eigen}
En lançant ces tests, j'ai remarqué qu'ils sont très long à compiler et à se lancer, de
plus, ils ne fonctionnent pas tous à tous les coups. Les tests vont un peu plus vite sur
une machine plus puissante, mais cela reste très long.

Pour que je puisse tester efficacement et rapidement, j'ai donc implémenté des tests de non-régression.
Pour ces tests, j'ai copié les fonctions actuelles dans un nouveau fichier et
j'ai suffixé ces fonctions par \emph{_old}.

J'ai ensuite implémenté les tests de non-régression pour toutes les opérations présentes
dans \emph{PacketMath.h}. Ces tests fonctionnent pour les architectures \emph{SSE}, \emph{AVX},
\emph{AVX2} et quelque opérations en \emph{AVX512}.

Mes tests de non-régression sont capables d'afficher le contenu des vecteurs lorsqu'il y a
une différence entre le résultat de la nouvelle version et celui de l'ancienne version
de l'opération.
Ils sont aussi capables d'afficher le contenu des vecteurs en binaire et de dire, sans
arrêter les tests, quand le \emph{code MIPP} appelle une fonction \MIPP sur un type qui
n'est pas encore supporté par \emph{MIPP}

\subsection{L'implémentation}

Dans un premier temps, j'ai implémenté seulement les opérations \emph{SSE}, \emph{AVX} et
\emph{AVX2}.
J'ai commencé par l'implémentation de l'opération \emph{pset1} qui remplie un vecteur avec
la même valeur à chaque case. Dans le \hyperref[eigenTypesTable]{tableau des types \Eigen},
on voit que les types \emph{SSE} font 128 bits et les types \emph{AVX} font 256 bits. Je
suis donc parti sur une mauvaise piste en convertissant les \emph{Packet4f},
\emph{Packet2d} et \emph{Packet4i} en \emph{reg2} et les \emph{Packet8f}, \emph{Packet4d}
et \emph{Packet8i} en \emph{reg}. Le problème est que \MIPP ne support qu'une seule taille
de vecteur à la fois donc il ne peut pas faire des opérations sur les \emph{reg2} mais
que sur les \emph{reg}. Ce problème m'a amené à devoir transformer mes vecteurs \emph{reg2}
en \emph{reg} ce qui se fait avec les fonctions \emph{combinate()} et \emph{low()}, pour
respectivement fusionner deux \emph{reg2} vecteurs en un vecteur \emph{reg} et récupérer
la première moitié \emph{reg2} d'un vecteur \emph{reg}. Toutes ces transformations amenaient
à deux principaux problèmes :  le premier est qu'on a un surcoût supplémentaire à chaque
appel. Le second est que cette stratégie ne fonctionne pas avec \emph{AVX512} qu'on a
mis de côté pour l'instant. Néanmoins, ces premières implémentations fonctionnent et passent
mes tests.

Pour palier ce problème, j'ai donc ajouté un système de conversion. Pour cela j'ai dû, pour
chaque architecture, ajouter des casts\footnote{conversion d'un type à un autre} simples et
des casts en intrinsic pour les changements de taille. Par exemple, un vecteur de 128 bits
que je dois convertir en vecteur de 512 bits et inversement.
Ces conversions peuvent amener à une perte de performance.

% TODO: (Benchmark here ?)

J'ai ensuite continué l'implémentation en \emph{code MIPP}, de plusieurs opérations pour
chaque type \emph{SSE} et \emph{AVX}, qu'elle supporte. Ce qui m'a amené à avoir plusieurs
fois la même implémentation \MIPP avec seulement le type qui change. Par exemple,
l'opération \emph{pset1<Packet4f>()} et l'opération \emph{pset1<Packet4i>}.

Dans un second temps, le but à été de \textit{replier} les fonctions pour en avoir qu'une
seule. Mais je suis tombé sur un problème, le comportement par défaut est une opération
scalaire. En effet, \Eigen permet d'activer ou non la vectorisation. Lorsque la
vectorisation et désactivée, ce sont les mêmes opérations qui sont appelées, mais dans ce cas,
ce sont les opérations par défaut qui sont appelées. Et dans le cas où la vectorisation est
activée, on tombe sur un cas spécifique, les cas avec les \emph{Packet} qui sont précisés
à l'appel. On se retrouve donc avec 2 cas génériques, un pour les scalaires
et un pour les vecteurs (\emph{Packet}). Cette situation amenait à un état compliqué qui
reste encore à résoudre.

Je vois deux solutions qui fonctionneraient peut-être :

\begin{itemize}
  \item finir de tout implémenter puis faire en sorte que tous les \emph{Packet}
  correspondent au type \emph{reg} de \MIPP ce qui permettrait de n'avoir qu'un seul cas
  spécifique qui correspondrait au \emph{code MIPP}. Cela nous permettrait en plus de ne
  plus avoir besoin du système de conversion.
  \item ajouter des adaptateurs d'objet pour avoir une arborescence et utiliser le
  polymorphisme pour avoir un seul cas spécifique qui correspondrait au \emph{code MIPP}.
\end{itemize}
